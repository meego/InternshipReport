
% Default to the notebook output style

    


% Inherit from the specified cell style.




    
\documentclass{article}

    
    
    \usepackage{graphicx} % Used to insert images
    \usepackage{adjustbox} % Used to constrain images to a maximum size 
    \usepackage{color} % Allow colors to be defined
    \usepackage{enumerate} % Needed for markdown enumerations to work
    \usepackage{geometry} % Used to adjust the document margins
    \usepackage{amsmath} % Equations
    \usepackage{amssymb} % Equations
    \usepackage[mathletters]{ucs} % Extended unicode (utf-8) support
    \usepackage[utf8x]{inputenc} % Allow utf-8 characters in the tex document
    \usepackage{fancyvrb} % verbatim replacement that allows latex
    \usepackage{grffile} % extends the file name processing of package graphics 
                         % to support a larger range 
    % The hyperref package gives us a pdf with properly built
    % internal navigation ('pdf bookmarks' for the table of contents,
    % internal cross-reference links, web links for URLs, etc.)
    \usepackage{hyperref}
    \usepackage{longtable} % longtable support required by pandoc >1.10
    \usepackage{booktabs}  % table support for pandoc > 1.12.2
    

    
    
    \definecolor{orange}{cmyk}{0,0.4,0.8,0.2}
    \definecolor{darkorange}{rgb}{.71,0.21,0.01}
    \definecolor{darkgreen}{rgb}{.12,.54,.11}
    \definecolor{myteal}{rgb}{.26, .44, .56}
    \definecolor{gray}{gray}{0.45}
    \definecolor{lightgray}{gray}{.95}
    \definecolor{mediumgray}{gray}{.8}
    \definecolor{inputbackground}{rgb}{.95, .95, .85}
    \definecolor{outputbackground}{rgb}{.95, .95, .95}
    \definecolor{traceback}{rgb}{1, .95, .95}
    % ansi colors
    \definecolor{red}{rgb}{.6,0,0}
    \definecolor{green}{rgb}{0,.65,0}
    \definecolor{brown}{rgb}{0.6,0.6,0}
    \definecolor{blue}{rgb}{0,.145,.698}
    \definecolor{purple}{rgb}{.698,.145,.698}
    \definecolor{cyan}{rgb}{0,.698,.698}
    \definecolor{lightgray}{gray}{0.5}
    
    % bright ansi colors
    \definecolor{darkgray}{gray}{0.25}
    \definecolor{lightred}{rgb}{1.0,0.39,0.28}
    \definecolor{lightgreen}{rgb}{0.48,0.99,0.0}
    \definecolor{lightblue}{rgb}{0.53,0.81,0.92}
    \definecolor{lightpurple}{rgb}{0.87,0.63,0.87}
    \definecolor{lightcyan}{rgb}{0.5,1.0,0.83}
    
    % commands and environments needed by pandoc snippets
    % extracted from the output of `pandoc -s`
    \DefineVerbatimEnvironment{Highlighting}{Verbatim}{commandchars=\\\{\}}
    % Add ',fontsize=\small' for more characters per line
    \newenvironment{Shaded}{}{}
    \newcommand{\KeywordTok}[1]{\textcolor[rgb]{0.00,0.44,0.13}{\textbf{{#1}}}}
    \newcommand{\DataTypeTok}[1]{\textcolor[rgb]{0.56,0.13,0.00}{{#1}}}
    \newcommand{\DecValTok}[1]{\textcolor[rgb]{0.25,0.63,0.44}{{#1}}}
    \newcommand{\BaseNTok}[1]{\textcolor[rgb]{0.25,0.63,0.44}{{#1}}}
    \newcommand{\FloatTok}[1]{\textcolor[rgb]{0.25,0.63,0.44}{{#1}}}
    \newcommand{\CharTok}[1]{\textcolor[rgb]{0.25,0.44,0.63}{{#1}}}
    \newcommand{\StringTok}[1]{\textcolor[rgb]{0.25,0.44,0.63}{{#1}}}
    \newcommand{\CommentTok}[1]{\textcolor[rgb]{0.38,0.63,0.69}{\textit{{#1}}}}
    \newcommand{\OtherTok}[1]{\textcolor[rgb]{0.00,0.44,0.13}{{#1}}}
    \newcommand{\AlertTok}[1]{\textcolor[rgb]{1.00,0.00,0.00}{\textbf{{#1}}}}
    \newcommand{\FunctionTok}[1]{\textcolor[rgb]{0.02,0.16,0.49}{{#1}}}
    \newcommand{\RegionMarkerTok}[1]{{#1}}
    \newcommand{\ErrorTok}[1]{\textcolor[rgb]{1.00,0.00,0.00}{\textbf{{#1}}}}
    \newcommand{\NormalTok}[1]{{#1}}
    
    % Define a nice break command that doesn't care if a line doesn't already
    % exist.
    \def\br{\hspace*{\fill} \\* }
    % Math Jax compatability definitions
    \def\gt{>}
    \def\lt{<}
    % Document parameters
    \title{FullRun}
    
    
    

    % Pygments definitions
    
\makeatletter
\def\PY@reset{\let\PY@it=\relax \let\PY@bf=\relax%
    \let\PY@ul=\relax \let\PY@tc=\relax%
    \let\PY@bc=\relax \let\PY@ff=\relax}
\def\PY@tok#1{\csname PY@tok@#1\endcsname}
\def\PY@toks#1+{\ifx\relax#1\empty\else%
    \PY@tok{#1}\expandafter\PY@toks\fi}
\def\PY@do#1{\PY@bc{\PY@tc{\PY@ul{%
    \PY@it{\PY@bf{\PY@ff{#1}}}}}}}
\def\PY#1#2{\PY@reset\PY@toks#1+\relax+\PY@do{#2}}

\expandafter\def\csname PY@tok@gd\endcsname{\def\PY@tc##1{\textcolor[rgb]{0.63,0.00,0.00}{##1}}}
\expandafter\def\csname PY@tok@gu\endcsname{\let\PY@bf=\textbf\def\PY@tc##1{\textcolor[rgb]{0.50,0.00,0.50}{##1}}}
\expandafter\def\csname PY@tok@gt\endcsname{\def\PY@tc##1{\textcolor[rgb]{0.00,0.27,0.87}{##1}}}
\expandafter\def\csname PY@tok@gs\endcsname{\let\PY@bf=\textbf}
\expandafter\def\csname PY@tok@gr\endcsname{\def\PY@tc##1{\textcolor[rgb]{1.00,0.00,0.00}{##1}}}
\expandafter\def\csname PY@tok@cm\endcsname{\let\PY@it=\textit\def\PY@tc##1{\textcolor[rgb]{0.25,0.50,0.50}{##1}}}
\expandafter\def\csname PY@tok@vg\endcsname{\def\PY@tc##1{\textcolor[rgb]{0.10,0.09,0.49}{##1}}}
\expandafter\def\csname PY@tok@m\endcsname{\def\PY@tc##1{\textcolor[rgb]{0.40,0.40,0.40}{##1}}}
\expandafter\def\csname PY@tok@mh\endcsname{\def\PY@tc##1{\textcolor[rgb]{0.40,0.40,0.40}{##1}}}
\expandafter\def\csname PY@tok@go\endcsname{\def\PY@tc##1{\textcolor[rgb]{0.53,0.53,0.53}{##1}}}
\expandafter\def\csname PY@tok@ge\endcsname{\let\PY@it=\textit}
\expandafter\def\csname PY@tok@vc\endcsname{\def\PY@tc##1{\textcolor[rgb]{0.10,0.09,0.49}{##1}}}
\expandafter\def\csname PY@tok@il\endcsname{\def\PY@tc##1{\textcolor[rgb]{0.40,0.40,0.40}{##1}}}
\expandafter\def\csname PY@tok@cs\endcsname{\let\PY@it=\textit\def\PY@tc##1{\textcolor[rgb]{0.25,0.50,0.50}{##1}}}
\expandafter\def\csname PY@tok@cp\endcsname{\def\PY@tc##1{\textcolor[rgb]{0.74,0.48,0.00}{##1}}}
\expandafter\def\csname PY@tok@gi\endcsname{\def\PY@tc##1{\textcolor[rgb]{0.00,0.63,0.00}{##1}}}
\expandafter\def\csname PY@tok@gh\endcsname{\let\PY@bf=\textbf\def\PY@tc##1{\textcolor[rgb]{0.00,0.00,0.50}{##1}}}
\expandafter\def\csname PY@tok@ni\endcsname{\let\PY@bf=\textbf\def\PY@tc##1{\textcolor[rgb]{0.60,0.60,0.60}{##1}}}
\expandafter\def\csname PY@tok@nl\endcsname{\def\PY@tc##1{\textcolor[rgb]{0.63,0.63,0.00}{##1}}}
\expandafter\def\csname PY@tok@nn\endcsname{\let\PY@bf=\textbf\def\PY@tc##1{\textcolor[rgb]{0.00,0.00,1.00}{##1}}}
\expandafter\def\csname PY@tok@no\endcsname{\def\PY@tc##1{\textcolor[rgb]{0.53,0.00,0.00}{##1}}}
\expandafter\def\csname PY@tok@na\endcsname{\def\PY@tc##1{\textcolor[rgb]{0.49,0.56,0.16}{##1}}}
\expandafter\def\csname PY@tok@nb\endcsname{\def\PY@tc##1{\textcolor[rgb]{0.00,0.50,0.00}{##1}}}
\expandafter\def\csname PY@tok@nc\endcsname{\let\PY@bf=\textbf\def\PY@tc##1{\textcolor[rgb]{0.00,0.00,1.00}{##1}}}
\expandafter\def\csname PY@tok@nd\endcsname{\def\PY@tc##1{\textcolor[rgb]{0.67,0.13,1.00}{##1}}}
\expandafter\def\csname PY@tok@ne\endcsname{\let\PY@bf=\textbf\def\PY@tc##1{\textcolor[rgb]{0.82,0.25,0.23}{##1}}}
\expandafter\def\csname PY@tok@nf\endcsname{\def\PY@tc##1{\textcolor[rgb]{0.00,0.00,1.00}{##1}}}
\expandafter\def\csname PY@tok@si\endcsname{\let\PY@bf=\textbf\def\PY@tc##1{\textcolor[rgb]{0.73,0.40,0.53}{##1}}}
\expandafter\def\csname PY@tok@s2\endcsname{\def\PY@tc##1{\textcolor[rgb]{0.73,0.13,0.13}{##1}}}
\expandafter\def\csname PY@tok@vi\endcsname{\def\PY@tc##1{\textcolor[rgb]{0.10,0.09,0.49}{##1}}}
\expandafter\def\csname PY@tok@nt\endcsname{\let\PY@bf=\textbf\def\PY@tc##1{\textcolor[rgb]{0.00,0.50,0.00}{##1}}}
\expandafter\def\csname PY@tok@nv\endcsname{\def\PY@tc##1{\textcolor[rgb]{0.10,0.09,0.49}{##1}}}
\expandafter\def\csname PY@tok@s1\endcsname{\def\PY@tc##1{\textcolor[rgb]{0.73,0.13,0.13}{##1}}}
\expandafter\def\csname PY@tok@sh\endcsname{\def\PY@tc##1{\textcolor[rgb]{0.73,0.13,0.13}{##1}}}
\expandafter\def\csname PY@tok@sc\endcsname{\def\PY@tc##1{\textcolor[rgb]{0.73,0.13,0.13}{##1}}}
\expandafter\def\csname PY@tok@sx\endcsname{\def\PY@tc##1{\textcolor[rgb]{0.00,0.50,0.00}{##1}}}
\expandafter\def\csname PY@tok@bp\endcsname{\def\PY@tc##1{\textcolor[rgb]{0.00,0.50,0.00}{##1}}}
\expandafter\def\csname PY@tok@c1\endcsname{\let\PY@it=\textit\def\PY@tc##1{\textcolor[rgb]{0.25,0.50,0.50}{##1}}}
\expandafter\def\csname PY@tok@kc\endcsname{\let\PY@bf=\textbf\def\PY@tc##1{\textcolor[rgb]{0.00,0.50,0.00}{##1}}}
\expandafter\def\csname PY@tok@c\endcsname{\let\PY@it=\textit\def\PY@tc##1{\textcolor[rgb]{0.25,0.50,0.50}{##1}}}
\expandafter\def\csname PY@tok@mf\endcsname{\def\PY@tc##1{\textcolor[rgb]{0.40,0.40,0.40}{##1}}}
\expandafter\def\csname PY@tok@err\endcsname{\def\PY@bc##1{\setlength{\fboxsep}{0pt}\fcolorbox[rgb]{1.00,0.00,0.00}{1,1,1}{\strut ##1}}}
\expandafter\def\csname PY@tok@kd\endcsname{\let\PY@bf=\textbf\def\PY@tc##1{\textcolor[rgb]{0.00,0.50,0.00}{##1}}}
\expandafter\def\csname PY@tok@ss\endcsname{\def\PY@tc##1{\textcolor[rgb]{0.10,0.09,0.49}{##1}}}
\expandafter\def\csname PY@tok@sr\endcsname{\def\PY@tc##1{\textcolor[rgb]{0.73,0.40,0.53}{##1}}}
\expandafter\def\csname PY@tok@mo\endcsname{\def\PY@tc##1{\textcolor[rgb]{0.40,0.40,0.40}{##1}}}
\expandafter\def\csname PY@tok@kn\endcsname{\let\PY@bf=\textbf\def\PY@tc##1{\textcolor[rgb]{0.00,0.50,0.00}{##1}}}
\expandafter\def\csname PY@tok@mi\endcsname{\def\PY@tc##1{\textcolor[rgb]{0.40,0.40,0.40}{##1}}}
\expandafter\def\csname PY@tok@gp\endcsname{\let\PY@bf=\textbf\def\PY@tc##1{\textcolor[rgb]{0.00,0.00,0.50}{##1}}}
\expandafter\def\csname PY@tok@o\endcsname{\def\PY@tc##1{\textcolor[rgb]{0.40,0.40,0.40}{##1}}}
\expandafter\def\csname PY@tok@kr\endcsname{\let\PY@bf=\textbf\def\PY@tc##1{\textcolor[rgb]{0.00,0.50,0.00}{##1}}}
\expandafter\def\csname PY@tok@s\endcsname{\def\PY@tc##1{\textcolor[rgb]{0.73,0.13,0.13}{##1}}}
\expandafter\def\csname PY@tok@kp\endcsname{\def\PY@tc##1{\textcolor[rgb]{0.00,0.50,0.00}{##1}}}
\expandafter\def\csname PY@tok@w\endcsname{\def\PY@tc##1{\textcolor[rgb]{0.73,0.73,0.73}{##1}}}
\expandafter\def\csname PY@tok@kt\endcsname{\def\PY@tc##1{\textcolor[rgb]{0.69,0.00,0.25}{##1}}}
\expandafter\def\csname PY@tok@ow\endcsname{\let\PY@bf=\textbf\def\PY@tc##1{\textcolor[rgb]{0.67,0.13,1.00}{##1}}}
\expandafter\def\csname PY@tok@sb\endcsname{\def\PY@tc##1{\textcolor[rgb]{0.73,0.13,0.13}{##1}}}
\expandafter\def\csname PY@tok@k\endcsname{\let\PY@bf=\textbf\def\PY@tc##1{\textcolor[rgb]{0.00,0.50,0.00}{##1}}}
\expandafter\def\csname PY@tok@se\endcsname{\let\PY@bf=\textbf\def\PY@tc##1{\textcolor[rgb]{0.73,0.40,0.13}{##1}}}
\expandafter\def\csname PY@tok@sd\endcsname{\let\PY@it=\textit\def\PY@tc##1{\textcolor[rgb]{0.73,0.13,0.13}{##1}}}

\def\PYZbs{\char`\\}
\def\PYZus{\char`\_}
\def\PYZob{\char`\{}
\def\PYZcb{\char`\}}
\def\PYZca{\char`\^}
\def\PYZam{\char`\&}
\def\PYZlt{\char`\<}
\def\PYZgt{\char`\>}
\def\PYZsh{\char`\#}
\def\PYZpc{\char`\%}
\def\PYZdl{\char`\$}
\def\PYZhy{\char`\-}
\def\PYZsq{\char`\'}
\def\PYZdq{\char`\"}
\def\PYZti{\char`\~}
% for compatibility with earlier versions
\def\PYZat{@}
\def\PYZlb{[}
\def\PYZrb{]}
\makeatother


    % Exact colors from NB
    \definecolor{incolor}{rgb}{0.0, 0.0, 0.5}
    \definecolor{outcolor}{rgb}{0.545, 0.0, 0.0}



    
    % Prevent overflowing lines due to hard-to-break entities
    \sloppy 
    % Setup hyperref package
    \hypersetup{
      breaklinks=true,  % so long urls are correctly broken across lines
      colorlinks=true,
      urlcolor=blue,
      linkcolor=darkorange,
      citecolor=darkgreen,
      }
    % Slightly bigger margins than the latex defaults
    
    \geometry{verbose,tmargin=1in,bmargin=1in,lmargin=1in,rmargin=1in}
    
    

    \begin{document}
    
    
    \maketitle
    
    

    

    \section{From Preprocessing to Evaluation}


    This notebook shows the different steps that are needed to classify the
brain cross sections. It starts with the data preprocessing and ends
with a final model. Several different functions are used by the imported
custom modules. This is done, so that the notebook stays clear. If you
want to take a detailed look at the used functions, feel free to use the
different source files.

At the beginning the client is set up, so that the engines can be
accessed and used.

    \begin{Verbatim}[commandchars=\\\{\}]
{\color{incolor}In [{\color{incolor}1}]:} \PY{c}{\PYZsh{} Import the client and the imp module to load source files on all engines}
        \PY{k+kn}{from} \PY{n+nn}{IPython.parallel} \PY{k+kn}{import} \PY{n}{Client}
        \PY{n}{c} \PY{o}{=} \PY{n}{Client}\PY{p}{(}\PY{p}{)}
        \PY{n}{dview} \PY{o}{=} \PY{n}{c}\PY{p}{[}\PY{p}{:}\PY{p}{]}
        \PY{n}{dview}\PY{o}{.}\PY{n}{block} \PY{o}{=} \PY{n+nb+bp}{True}
        \PY{k}{with} \PY{n}{dview}\PY{o}{.}\PY{n}{sync\PYZus{}imports}\PY{p}{(}\PY{p}{)}\PY{p}{:}
            \PY{k+kn}{import} \PY{n+nn}{imp}
\end{Verbatim}

    \begin{Verbatim}[commandchars=\\\{\}]
importing imp on engine(s)
    \end{Verbatim}


    \paragraph{Generating samples}


    The training data can be sampled. In this example this is done by
rescaling some images to a size of 256x256.

    \begin{Verbatim}[commandchars=\\\{\}]
{\color{incolor}In [{\color{incolor}2}]:} \PY{c}{\PYZsh{} resizing the images}
        \PY{k+kn}{import} \PY{n+nn}{glob}
        \PY{n}{masks} \PY{o}{=} \PY{n}{glob}\PY{o}{.}\PY{n}{glob}\PY{p}{(}\PY{l+s}{\PYZdq{}}\PY{l+s}{../classification/data/rescaled256x256/masks/MSA*.tif}\PY{l+s}{\PYZdq{}}\PY{p}{)}
        \PY{n}{images} \PY{o}{=} \PY{n}{glob}\PY{o}{.}\PY{n}{glob}\PY{p}{(}\PY{l+s}{\PYZdq{}}\PY{l+s}{../classification/data/rescaled256x256/images/MSA*.tif}\PY{l+s}{\PYZdq{}}\PY{p}{)}
        
        \PY{k+kn}{from} \PY{n+nn}{PIL} \PY{k+kn}{import} \PY{n}{Image}
        
        \PY{n}{size}\PY{o}{=}\PY{p}{(}\PY{l+m+mi}{256}\PY{p}{,}\PY{l+m+mi}{256}\PY{p}{)}
        \PY{n}{resized\PYZus{}images} \PY{o}{=} \PY{p}{[}\PY{n}{Image}\PY{o}{.}\PY{n}{open}\PY{p}{(}\PY{n}{i}\PY{p}{)}\PY{o}{.}\PY{n}{resize}\PY{p}{(}\PY{n}{size}\PY{p}{)} \PY{k}{for} \PY{n}{i} \PY{o+ow}{in} \PY{n}{images}\PY{p}{]}
        \PY{n}{resized\PYZus{}masks} \PY{o}{=} \PY{p}{[}\PY{n}{Image}\PY{o}{.}\PY{n}{open}\PY{p}{(}\PY{n}{m}\PY{p}{)}\PY{o}{.}\PY{n}{resize}\PY{p}{(}\PY{n}{size}\PY{p}{)} \PY{k}{for} \PY{n}{m} \PY{o+ow}{in} \PY{n}{images}\PY{p}{]}
\end{Verbatim}


    \paragraph{Converting and Adding Features}


    The following code block shows how to generate libsvm formated data from
an original image and the hand labeled mask. This is done on multiple
engines started by ipython. In this example the used features are:

\begin{verbatim}
R,G,B,std(R),std(G),std(B),segmentation_bit,H,S,V
\end{verbatim}

The window size for the image is 16x16 pixel. The thresholds for
watershed segmentation are set as p=0.2 and q=0.8. Custom features can
be added easily. A feature that needs the original image needs to be set
in the new\_features list. If only the pixel is needed, the
online\_feature list is sufficient.

Because callables are passed, the user can add any new function. The
only condition is that the first argument is the image/pixel. If the
function has more arguments, the \emph{partial} function can be used to
create a new one with only one argument.

    \begin{Verbatim}[commandchars=\\\{\}]
{\color{incolor}In [{\color{incolor}3}]:} \PY{c}{\PYZsh{} load data conversion and feature extration modules locally and on the engines}
        \PY{n}{dc} \PY{o}{=} \PY{n}{imp}\PY{o}{.}\PY{n}{load\PYZus{}source}\PY{p}{(}\PY{l+s}{\PYZdq{}}\PY{l+s}{data\PYZus{}conversion}\PY{l+s}{\PYZdq{}}\PY{p}{,}\PYZbs{}
                             \PY{l+s}{\PYZdq{}}\PY{l+s}{../parallel/data\PYZus{}generation/data\PYZus{}conversion.py}\PY{l+s}{\PYZdq{}}\PY{p}{)}
        \PY{n}{fe} \PY{o}{=} \PY{n}{imp}\PY{o}{.}\PY{n}{load\PYZus{}source}\PY{p}{(}\PY{l+s}{\PYZdq{}}\PY{l+s}{feature\PYZus{}extraction}\PY{l+s}{\PYZdq{}}\PY{p}{,}\PYZbs{}
                             \PY{l+s}{\PYZdq{}}\PY{l+s}{../classification/feature\PYZus{}extraction.py}\PY{l+s}{\PYZdq{}}\PY{p}{)}
        \PY{c}{\PYZsh{} use the direct view to load the modules}
        \PY{n}{dview}\PY{o}{.}\PY{n}{execute}\PY{p}{(}\PY{l+s}{\PYZsq{}}\PY{l+s}{dc = imp.load\PYZus{}source(}\PY{l+s}{\PYZdq{}}\PY{l+s}{data\PYZus{}conversion}\PY{l+s}{\PYZdq{}}\PY{l+s}{,}\PY{l+s+se}{\PYZbs{}}
        \PY{l+s}{    }\PY{l+s}{\PYZdq{}}\PY{l+s}{../parallel/data\PYZus{}generation/data\PYZus{}conversion.py}\PY{l+s}{\PYZdq{}}\PY{l+s}{)}\PY{l+s}{\PYZsq{}}\PY{p}{)}
        \PY{n}{dview}\PY{o}{.}\PY{n}{execute}\PY{p}{(}\PY{l+s}{\PYZsq{}}\PY{l+s}{fe = imp.load\PYZus{}source(}\PY{l+s}{\PYZdq{}}\PY{l+s}{feature\PYZus{}extraction}\PY{l+s}{\PYZdq{}}\PY{l+s}{,}\PY{l+s+se}{\PYZbs{}}
        \PY{l+s}{    }\PY{l+s}{\PYZdq{}}\PY{l+s}{../classification/feature\PYZus{}extraction.py}\PY{l+s}{\PYZdq{}}\PY{l+s}{)}\PY{l+s}{\PYZsq{}}\PY{p}{)}
        
        \PY{c}{\PYZsh{} set masks and images, that are to be converted, as lists}
        \PY{c}{\PYZsh{} be aware that masks[0] must belong to images[0]. This is done by sorting the lists}
        \PY{k+kn}{import} \PY{n+nn}{glob}
        \PY{n}{masks} \PY{o}{=} \PY{n+nb}{sorted}\PY{p}{(}\PY{n}{glob}\PY{o}{.}\PY{n}{glob}\PY{p}{(}\PY{l+s}{\PYZdq{}}\PY{l+s}{../classification/data/rescaled256x256/masks/MSA*.tif}\PY{l+s}{\PYZdq{}}\PY{p}{)}\PY{p}{)}
        \PY{n}{images} \PY{o}{=} \PY{n+nb}{sorted}\PY{p}{(}\PY{n}{glob}\PY{o}{.}\PY{n}{glob}\PY{p}{(}\PY{l+s}{\PYZdq{}}\PY{l+s}{../classification/data/rescaled256x256/images/MSA*.tif}\PY{l+s}{\PYZdq{}}\PY{p}{)}\PY{p}{)}
        
        \PY{c}{\PYZsh{} scatter the images and mask to the engines}
        \PY{n}{dview}\PY{o}{.}\PY{n}{scatter}\PY{p}{(}\PY{l+s}{\PYZdq{}}\PY{l+s}{images}\PY{l+s}{\PYZdq{}}\PY{p}{,}\PY{n}{images}\PY{p}{)}
        \PY{n}{dview}\PY{o}{.}\PY{n}{scatter}\PY{p}{(}\PY{l+s}{\PYZdq{}}\PY{l+s}{masks}\PY{l+s}{\PYZdq{}}\PY{p}{,}\PY{n}{masks}\PY{p}{)}
        
        \PY{k}{with} \PY{n}{dview}\PY{o}{.}\PY{n}{sync\PYZus{}imports}\PY{p}{(}\PY{p}{)}\PY{p}{:}
            \PY{k+kn}{from} \PY{n+nn}{functools} \PY{k+kn}{import} \PY{n}{partial}
        
        \PY{c}{\PYZsh{} add features and convert the images}
        \PY{n}{out\PYZus{}dir} \PY{o}{=} \PY{l+s}{\PYZdq{}}\PY{l+s}{../classification/data/rescaled256x256/test}\PY{l+s}{\PYZdq{}}
        \PY{n}{dview}\PY{p}{[}\PY{l+s}{\PYZsq{}}\PY{l+s}{out\PYZus{}dir}\PY{l+s}{\PYZsq{}}\PY{p}{]} \PY{o}{=} \PY{n}{out\PYZus{}dir}
        \PY{n}{cmd} \PY{o}{=} \PY{l+s}{\PYZsq{}}\PY{l+s}{dc.convert\PYZus{}and\PYZus{}save(images,masks,out\PYZus{}dir,}\PY{l+s+se}{\PYZbs{}}
        \PY{l+s}{    new\PYZus{}features=[partial(fe.calc\PYZus{}std,size=16),}\PY{l+s+se}{\PYZbs{}}
        \PY{l+s}{    partial(fe.findSegmentation,p=0.2)],online\PYZus{}features=[fe.add\PYZus{}hsv])}\PY{l+s}{\PYZsq{}}
        \PY{n}{dview}\PY{o}{.}\PY{n}{execute}\PY{p}{(}\PY{n}{cmd}\PY{p}{)}
\end{Verbatim}

    \begin{Verbatim}[commandchars=\\\{\}]
importing partial from functools on engine(s)
    \end{Verbatim}

            \begin{Verbatim}[commandchars=\\\{\}]
{\color{outcolor}Out[{\color{outcolor}3}]:} <AsyncResult: finished>
\end{Verbatim}
        

    \paragraph{Create training and testing sets}


    The data can be combined into training and testing sets. The data is
shuffled and the sets are class balanced by default. The
\emph{combine\_files} method can be used to generate training sets for
the different layers, if multiple models are used.

    \begin{Verbatim}[commandchars=\\\{\}]
{\color{incolor}In [{\color{incolor}4}]:} \PY{n}{sampling} \PY{o}{=} \PY{n}{imp}\PY{o}{.}\PY{n}{load\PYZus{}source}\PY{p}{(}\PY{l+s}{\PYZdq{}}\PY{l+s}{sample\PYZus{}files}\PY{l+s}{\PYZdq{}}\PY{p}{,}\PY{l+s}{\PYZdq{}}\PY{l+s}{../data\PYZus{}generation/random\PYZus{}sampling/sample\PYZus{}files.py}\PY{l+s}{\PYZdq{}}\PY{p}{)}
        \PY{n}{sampling}\PY{o}{.}\PY{n}{OUT} \PY{o}{=} \PY{l+s}{\PYZdq{}}\PY{l+s}{../classification/data/rescaled256x256/test}\PY{l+s}{\PYZdq{}}
        
        \PY{n}{sampling}\PY{o}{.}\PY{n}{combine\PYZus{}files}\PY{p}{(}\PY{l+s}{\PYZdq{}}\PY{l+s}{../classification/data/rescaled256x256/test/}\PY{l+s}{\PYZdq{}}\PY{p}{,}\PY{l+s}{\PYZdq{}}\PY{l+s}{ALL.svm}\PY{l+s}{\PYZdq{}}\PY{p}{)}
        \PY{n}{sampling}\PY{o}{.}\PY{n}{split\PYZus{}train\PYZus{}test}\PY{p}{(}\PY{l+s}{\PYZdq{}}\PY{l+s}{../classification/data/rescaled256x256/test/}\PY{l+s}{\PYZdq{}}\PY{p}{,}\PYZbs{}
                                  \PY{l+s}{\PYZdq{}}\PY{l+s}{ALL.svm}\PY{l+s}{\PYZdq{}}\PY{p}{,}\PY{n}{test\PYZus{}size}\PY{o}{=}\PY{l+m+mf}{0.5}\PY{p}{,}\PY{n}{shuffle}\PY{o}{=}\PY{n+nb+bp}{True}\PY{p}{)}
\end{Verbatim}


    \paragraph{Build a SVM classifier}


    To build a classifier, first the training and test data have to be
loaded. In this case it is a single cross section, but any combination
of the different images can be used. This can be done with
\emph{sample.combine\_files}.

The data is normalized and additional features are added. Instead of the
RBFSampler the Nystroem method can be used to generate different
features or approximate another kernel.

At the end the classifier is trained with parameters optimized with a
grid search. Instead of building one classifier, grid search may be used
to test different parameters on the current data set. After this the
model can easily be saved with the pickle module which is part of the
python standard library. It can also be used to save the RBFSampler
which is needed if new data should be predicted.

    \begin{Verbatim}[commandchars=\\\{\}]
{\color{incolor}In [{\color{incolor}5}]:} \PY{k+kn}{from} \PY{n+nn}{skimage.io} \PY{k+kn}{import} \PY{n}{imshow}\PY{p}{,}\PY{n}{imread}
        \PY{k+kn}{import} \PY{n+nn}{numpy} \PY{k+kn}{as} \PY{n+nn}{np}
        \PY{k+kn}{from} \PY{n+nn}{sklearn.kernel\PYZus{}approximation} \PY{k+kn}{import} \PY{n}{RBFSampler}
        \PY{k+kn}{from} \PY{n+nn}{sklearn.svm} \PY{k+kn}{import} \PY{n}{LinearSVC}
        \PY{k+kn}{from} \PY{n+nn}{sklearn.datasets} \PY{k+kn}{import} \PY{n}{load\PYZus{}svmlight\PYZus{}file}
        
        \PY{c}{\PYZsh{} load training set}
        \PY{n}{X\PYZus{}train}\PY{p}{,}\PY{n}{y\PYZus{}train} \PY{o}{=} \PY{n}{load\PYZus{}svmlight\PYZus{}file}\PY{p}{(}\PYZbs{}
                        \PY{l+s}{\PYZdq{}}\PY{l+s}{../classification/data/rescaled256x256/MSA\PYZus{}03\PYZhy{}2009\PYZus{}dXXXX\PYZhy{}XX\PYZhy{}XX\PYZus{}s0110.svm}\PY{l+s}{\PYZdq{}}\PY{p}{)}
        \PY{n}{X\PYZus{}train} \PY{o}{=} \PY{n}{X\PYZus{}train}\PY{o}{.}\PY{n}{toarray}\PY{p}{(}\PY{p}{)}
        
        \PY{c}{\PYZsh{} normalize data}
        \PY{n}{max\PYZus{}val} \PY{o}{=} \PY{n}{X\PYZus{}train}\PY{o}{.}\PY{n}{max}\PY{p}{(}\PY{n}{axis}\PY{o}{=}\PY{l+m+mi}{0}\PY{p}{)}
        \PY{n}{min\PYZus{}val} \PY{o}{=} \PY{n}{X\PYZus{}train}\PY{o}{.}\PY{n}{min}\PY{p}{(}\PY{n}{axis}\PY{o}{=}\PY{l+m+mi}{0}\PY{p}{)}
        \PY{n}{X\PYZus{}train} \PY{o}{=} \PY{p}{(}\PY{n}{X\PYZus{}train}\PY{o}{\PYZhy{}}\PY{n}{min\PYZus{}val}\PY{p}{)}\PY{o}{/}\PY{p}{(}\PY{n}{max\PYZus{}val}\PY{o}{\PYZhy{}}\PY{n}{min\PYZus{}val}\PY{p}{)}
        
        \PY{c}{\PYZsh{} load test image}
        \PY{n}{X\PYZus{}test}\PY{p}{,}\PY{n}{y\PYZus{}test} \PY{o}{=} \PY{n}{load\PYZus{}svmlight\PYZus{}file}\PY{p}{(}\PYZbs{}
                        \PY{l+s}{\PYZdq{}}\PY{l+s}{../classification/data/segmentation/MSA\PYZus{}03\PYZhy{}2009\PYZus{}dXXXX\PYZhy{}XX\PYZhy{}XX\PYZus{}s0200.svm}\PY{l+s}{\PYZdq{}}\PY{p}{)}
        \PY{n}{X\PYZus{}test} \PY{o}{=} \PY{n}{X\PYZus{}test}\PY{o}{.}\PY{n}{toarray}\PY{p}{(}\PY{p}{)}
        \PY{n}{X\PYZus{}test} \PY{o}{=} \PY{p}{(}\PY{n}{X\PYZus{}test}\PY{o}{\PYZhy{}}\PY{n}{min\PYZus{}val}\PY{p}{)}\PY{o}{/}\PY{p}{(}\PY{n}{max\PYZus{}val}\PY{o}{\PYZhy{}}\PY{n}{min\PYZus{}val}\PY{p}{)}
        
        \PY{c}{\PYZsh{} generate additional features}
        \PY{n}{rbf} \PY{o}{=} \PY{n}{RBFSampler}\PY{p}{(}\PY{n}{gamma}\PY{o}{=}\PY{l+m+mi}{2}\PY{p}{)}
        \PY{n}{X\PYZus{}features} \PY{o}{=} \PY{n}{rbf}\PY{o}{.}\PY{n}{fit\PYZus{}transform}\PY{p}{(}\PY{n}{X\PYZus{}train}\PY{p}{)}
        \PY{n}{X\PYZus{}test\PYZus{}features} \PY{o}{=} \PY{n}{rbf}\PY{o}{.}\PY{n}{transform}\PY{p}{(}\PY{n}{X\PYZus{}test}\PY{p}{)}
        
        \PY{c}{\PYZsh{} train linear SVM on highdimensional data}
        \PY{n}{linearSVM} \PY{o}{=} \PY{n}{LinearSVC}\PY{p}{(}\PY{n}{dual}\PY{o}{=}\PY{n+nb+bp}{False}\PY{p}{,}\PY{n}{C}\PY{o}{=}\PY{l+m+mi}{1000000}\PY{p}{)}
        \PY{n}{linearSVM}\PY{o}{.}\PY{n}{fit}\PY{p}{(}\PY{n}{X\PYZus{}features}\PY{p}{,}\PY{n}{y\PYZus{}train}\PY{p}{)}
        
        \PY{c}{\PYZsh{} save model }
        \PY{k+kn}{import} \PY{n+nn}{pickle}
        \PY{k}{with} \PY{n+nb}{open}\PY{p}{(}\PY{l+s}{\PYZdq{}}\PY{l+s}{SVMClassifier.pkl}\PY{l+s}{\PYZdq{}}\PY{p}{,}\PY{l+s}{\PYZdq{}}\PY{l+s}{w}\PY{l+s}{\PYZdq{}}\PY{p}{)} \PY{k}{as} \PY{n}{f}\PY{p}{:}
            \PY{n}{pickle}\PY{o}{.}\PY{n}{dump}\PY{p}{(}\PY{n}{linearSVM}\PY{p}{,}\PY{n}{f}\PY{p}{)}
\end{Verbatim}


    \paragraph{Classify test image}


    If the classifier is not trained in the current session, it can be
loaded with the pickle module. After adding the random features the
class can be predicted usign the \emph{predict} method of the
classifier. The predicted labels can be reshaped, so that the new mask
can be printed.

    \begin{Verbatim}[commandchars=\\\{\}]
{\color{incolor}In [{\color{incolor}6}]:} \PY{c}{\PYZsh{} load model}
        \PY{k+kn}{import} \PY{n+nn}{pickle}
        \PY{k}{with} \PY{n+nb}{open}\PY{p}{(}\PY{l+s}{\PYZdq{}}\PY{l+s}{SVMClassifier.pkl}\PY{l+s}{\PYZdq{}}\PY{p}{,}\PY{l+s}{\PYZdq{}}\PY{l+s}{r}\PY{l+s}{\PYZdq{}}\PY{p}{)} \PY{k}{as} \PY{n}{f}\PY{p}{:}
            \PY{n}{linearSVM} \PY{o}{=} \PY{n}{pickle}\PY{o}{.}\PY{n}{load}\PY{p}{(}\PY{n}{f}\PY{p}{)}
            
        \PY{c}{\PYZsh{} load hand labeled mask}
        \PY{n}{mask} \PY{o}{=} \PY{n}{imread}\PY{p}{(}\PY{l+s}{\PYZdq{}}\PY{l+s}{../classification/data/segmentation/masks/MSA\PYZus{}03\PYZhy{}2009\PYZus{}dXXXX\PYZhy{}XX\PYZhy{}XX\PYZus{}s0200.tif}\PY{l+s}{\PYZdq{}}\PY{p}{)}
        
        \PY{n}{labels} \PY{o}{=} \PY{n}{linearSVM}\PY{o}{.}\PY{n}{predict}\PY{p}{(}\PY{n}{X\PYZus{}test\PYZus{}features}\PY{p}{)}
        \PY{n}{labels} \PY{o}{=} \PY{n}{labels}\PY{o}{.}\PY{n}{reshape}\PY{p}{(}\PY{n}{mask}\PY{o}{.}\PY{n}{shape}\PY{p}{)}
        
        \PY{o}{\PYZpc{}}\PY{k}{matplotlib} \PY{n}{inline}
        \PY{n}{imshow}\PY{p}{(}\PY{n}{labels}\PY{p}{)}
\end{Verbatim}

    \begin{center}
    \adjustimage{max size={0.9\linewidth}{0.9\paperheight}}{FullRun_files/FullRun_17_0.png}
    \end{center}
    { \hspace*{\fill} \\}
    
    If the data is also labeled by hand, the quality of the model can be
measured. The following function calculates the accuracy as well as the
f-score. It also returns the confusion matrix.

Because the classified data is an image, the confusion matrix is
visualized as an image. All true positives are colored white and all
true negatives black. With 100\% accuracy this would result in the hand
labeled mask. Since this is most often not the case, all false positives
are marked red and all false negatives blue.

This gives a good first impression on the quality of the classifier. If
the predicted labels are post processed by hand, it furthermore gives an
indication which regions cause problems.

    \begin{Verbatim}[commandchars=\\\{\}]
{\color{incolor}In [{\color{incolor}7}]:} \PY{k+kn}{from} \PY{n+nn}{sklearn.metrics} \PY{k+kn}{import} \PY{n}{confusion\PYZus{}matrix}\PY{p}{,}\PY{n}{f1\PYZus{}score}\PY{p}{,}\PY{n}{accuracy\PYZus{}score}
        \PY{k+kn}{from} \PY{n+nn}{collections} \PY{k+kn}{import} \PY{n}{namedtuple}
        
        \PY{n}{AccuracyMetrics} \PY{o}{=} \PY{n}{namedtuple}\PY{p}{(}\PY{l+s}{\PYZsq{}}\PY{l+s}{AccuracyMetrics}\PY{l+s}{\PYZsq{}}\PY{p}{,}\PYZbs{}
                                     \PY{p}{[}\PY{l+s}{\PYZsq{}}\PY{l+s}{accuracy}\PY{l+s}{\PYZsq{}}\PY{p}{,}\PY{l+s}{\PYZsq{}}\PY{l+s}{fscore}\PY{l+s}{\PYZsq{}}\PY{p}{,}\PY{l+s}{\PYZsq{}}\PY{l+s}{confusion\PYZus{}matrix}\PY{l+s}{\PYZsq{}}\PY{p}{,}\PY{l+s}{\PYZsq{}}\PY{l+s}{image}\PY{l+s}{\PYZsq{}}\PY{p}{]}\PY{p}{)}
        
        \PY{k}{def} \PY{n+nf}{compare\PYZus{}to\PYZus{}mask}\PY{p}{(}\PY{n}{pred}\PY{p}{,}\PY{n}{mask}\PY{p}{)}\PY{p}{:}
            \PY{n}{size} \PY{o}{=} \PY{n}{pred}\PY{o}{.}\PY{n}{shape}
            \PY{n}{img} \PY{o}{=} \PY{n}{np}\PY{o}{.}\PY{n}{zeros}\PY{p}{(}\PY{p}{(}\PY{n}{mask}\PY{o}{.}\PY{n}{shape}\PY{p}{[}\PY{l+m+mi}{0}\PY{p}{]}\PY{p}{,}\PY{n}{mask}\PY{o}{.}\PY{n}{shape}\PY{p}{[}\PY{l+m+mi}{1}\PY{p}{]}\PY{p}{,}\PY{l+m+mi}{3}\PY{p}{)}\PY{p}{,}\PY{n}{dtype}\PY{o}{=}\PY{l+s}{\PYZsq{}}\PY{l+s}{uint8}\PY{l+s}{\PYZsq{}}\PY{p}{)}
            \PY{n}{tp} \PY{o}{=} \PY{n}{np}\PY{o}{.}\PY{n}{dstack}\PY{p}{(}\PY{p}{(}\PY{p}{(}\PY{n}{pred}\PY{o}{\PYZgt{}}\PY{l+m+mi}{0}\PY{p}{)}\PY{o}{.}\PY{n}{reshape}\PY{p}{(}\PY{n}{size}\PY{p}{)}\PY{p}{,}\PY{p}{(}\PY{n}{mask}\PY{o}{\PYZgt{}}\PY{l+m+mi}{0}\PY{p}{)}\PY{o}{.}\PY{n}{reshape}\PY{p}{(}\PY{n}{size}\PY{p}{)}\PY{p}{)}\PY{p}{)}
            \PY{n}{fp} \PY{o}{=} \PY{n}{np}\PY{o}{.}\PY{n}{dstack}\PY{p}{(}\PY{p}{(}\PY{p}{(}\PY{n}{pred}\PY{o}{\PYZgt{}}\PY{l+m+mi}{0}\PY{p}{)}\PY{o}{.}\PY{n}{reshape}\PY{p}{(}\PY{n}{size}\PY{p}{)}\PY{p}{,}\PY{p}{(}\PY{n}{mask}\PY{o}{==}\PY{l+m+mi}{0}\PY{p}{)}\PY{o}{.}\PY{n}{reshape}\PY{p}{(}\PY{n}{size}\PY{p}{)}\PY{p}{)}\PY{p}{)}
            \PY{n}{fn} \PY{o}{=} \PY{n}{np}\PY{o}{.}\PY{n}{dstack}\PY{p}{(}\PY{p}{(}\PY{p}{(}\PY{n}{pred}\PY{o}{==}\PY{l+m+mi}{0}\PY{p}{)}\PY{o}{.}\PY{n}{reshape}\PY{p}{(}\PY{n}{size}\PY{p}{)}\PY{p}{,}\PY{p}{(}\PY{n}{mask}\PY{o}{\PYZgt{}}\PY{l+m+mi}{0}\PY{p}{)}\PY{o}{.}\PY{n}{reshape}\PY{p}{(}\PY{n}{size}\PY{p}{)}\PY{p}{)}\PY{p}{)}
            \PY{n}{img}\PY{p}{[}\PY{n}{np}\PY{o}{.}\PY{n}{all}\PY{p}{(}\PY{n}{tp}\PY{p}{,}\PY{n}{axis}\PY{o}{=}\PY{l+m+mi}{2}\PY{p}{)}\PY{p}{,}\PY{p}{:}\PY{p}{]} \PY{o}{=} \PY{l+m+mi}{255}
            \PY{n}{img}\PY{p}{[}\PY{n}{np}\PY{o}{.}\PY{n}{all}\PY{p}{(}\PY{n}{fp}\PY{p}{,}\PY{n}{axis}\PY{o}{=}\PY{l+m+mi}{2}\PY{p}{)}\PY{p}{,}\PY{l+m+mi}{0}\PY{p}{]} \PY{o}{=} \PY{l+m+mi}{255}
            \PY{n}{img}\PY{p}{[}\PY{n}{np}\PY{o}{.}\PY{n}{all}\PY{p}{(}\PY{n}{fn}\PY{p}{,}\PY{n}{axis}\PY{o}{=}\PY{l+m+mi}{2}\PY{p}{)}\PY{p}{,}\PY{l+m+mi}{2}\PY{p}{]} \PY{o}{=} \PY{l+m+mi}{255}
            
            \PY{n}{acc} \PY{o}{=} \PY{n}{accuracy\PYZus{}score}\PY{p}{(}\PY{n}{mask}\PY{o}{.}\PY{n}{flatten}\PY{p}{(}\PY{p}{)}\PY{p}{,}\PY{n}{pred}\PY{o}{.}\PY{n}{flatten}\PY{p}{(}\PY{p}{)}\PY{p}{,}\PY{n}{normalize}\PY{o}{=}\PY{n+nb+bp}{True}\PY{p}{)}
            \PY{n}{fscore} \PY{o}{=} \PY{n}{f1\PYZus{}score}\PY{p}{(}\PY{n}{mask}\PY{o}{.}\PY{n}{flatten}\PY{p}{(}\PY{p}{)}\PY{p}{,}\PY{n}{pred}\PY{o}{.}\PY{n}{flatten}\PY{p}{(}\PY{p}{)}\PY{p}{)}
            \PY{n}{cm} \PY{o}{=} \PY{n}{confusion\PYZus{}matrix}\PY{p}{(}\PY{n}{mask}\PY{o}{.}\PY{n}{flatten}\PY{p}{(}\PY{p}{)}\PY{p}{,}\PY{n}{pred}\PY{o}{.}\PY{n}{flatten}\PY{p}{(}\PY{p}{)}\PY{p}{)}
            
            \PY{k}{return} \PY{n}{AccuracyMetrics}\PY{p}{(}\PY{n}{acc}\PY{p}{,}\PY{n}{fscore}\PY{p}{,}\PY{n}{cm}\PY{p}{,}\PY{n}{img}\PY{p}{)}
        
        \PY{n}{metrics} \PY{o}{=} \PY{n}{compare\PYZus{}to\PYZus{}mask}\PY{p}{(}\PY{n}{labels}\PY{p}{,}\PY{n}{y\PYZus{}test}\PY{o}{.}\PY{n}{reshape}\PY{p}{(}\PY{n}{labels}\PY{o}{.}\PY{n}{shape}\PY{p}{)}\PY{p}{)}
        \PY{n}{imshow}\PY{p}{(}\PY{n}{metrics}\PY{p}{[}\PY{l+m+mi}{3}\PY{p}{]}\PY{p}{)}
\end{Verbatim}

    \begin{center}
    \adjustimage{max size={0.9\linewidth}{0.9\paperheight}}{FullRun_files/FullRun_19_0.png}
    \end{center}
    { \hspace*{\fill} \\}
    

    % Add a bibliography block to the postdoc
    
    
    
    \end{document}
