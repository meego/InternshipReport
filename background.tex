
\section{Background}

The primary programming language used during this internship is python.
Python is a dynamic scripting language and provides many nice modules
for scientific use cases like numpy and scipy. There are also some
machine learning modules and the scikit-learn module \cite{sklearn}
is one of the most used and stable ones. While python alone may not
have the best performance, it is easy to improve it by using c or
fortran code. This is done by most libraries. The serial support vector
machine algorithms of scikit-learn are based on the well known libsvm
library for example. This makes python and the scikit-learn module
a good choice for classification tasks. 

The tasks are executed on a super computer at the Juelich Supercomputing
Centre. It uses a PBS based batch system to handle the different jobs
of users and distribute the available nodes. A popular method for
distributed memory multiprocessing is MPI. It is, however, not being
used here.

IPython is an interactive python shell with many improvements. One
of them is interactive parallel computing. This allows parallelizing
python code in an interactive python session and has a robust error
handling. 

\lstinputlisting[language=bash,caption=pbs.controller.template,label=pbs_template]{./code/pbs.engine.template}

After setting up a profile and creating PBS templates (e.g. listing
\ref{pbs_template}), you can start a cluster easily using:

\[
ipcluster~start~--profile=pbs~-n~8
\]
 

More information on ipython in general and the setup for parallel
usage can be found at \cite{ipython_doc}.
